\abstract{
Genomic structural variations are an important class of genetic variants with a wide variety of functional impacts. The detection of structural variations using high-throughput short-read sequencing data is a difficult problem, and published algorithms do not provide the sensitivity and specificity required in research and clinical settings. Meanwhile, high-throughput sequencing is rapidly generating ever-larger data sets, necessitating the development of algorithms that can provide results rapidly and scale to use cloud and cluster infrastructures. MapReduce and Hadoop are becoming a standard for managing the distributed processing of large data sets, but existing structural variation detection approaches are difficult to translate into the MapReduce framework. We have formulated a general framework for structural variation detection in MapReduce, and implemented a software package called Cloudbreak, which detects genomic deletions and insertions with very high accuracy compared to existing popular tools. Through the use of MapReduce and Hadoop, Cloudbreak can scale to harness large compute clusters and big data sets, leading to much faster runtimes than existing methods. In addition, we show that Cloudbreak's formulation of the structural variation detection problem in terms of local feature generation allows it to simultaneously integrate many informative signals in statistical learning frameworks. We demonstrate this using conditional random fields, which enable learning conditional probability distributions over labels on sequences of observations, and show that it improves Cloudbreak's results, in particular increasing breakpoint resolution. 

In addition to the development of Cloudbreak and its extensions, we describe a data analysis project in which we examined the genomic features that occur near evolutionary breakpoints in the genome of the gibbon, whose karyotype is heavily rearranged compared to other primate species. Using a distributed pipeline to conduct Monte Carlo permutation tests, we find a statistical enrichment of segmental duplications, certain families of transposable elements, and evolutionarily shared binding sites of the protein CTCF near the locations of gibbon rearrangements. These findings may help us understand the process by which structural variations formed and were preserved in the gibbon lineage.
}

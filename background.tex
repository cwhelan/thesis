\chapter{Biological Background}\label{chap_background}

In this chapter we will explore the nature of structural variations, discuss the mechanisms that may be behind their formation, and their effect on phenotypes and disease. We will then describe high-throughput sequencing and discuss the typical informatics pipelines that researchers create to analyze sequencing data, placing SV detection within the context of the other types of compution that must be done to support end-to-end analysis. Finally, we will discus the challenges that the increasing number and size of sequencing data sets represent.

\section{Structural Variations}

As we described in the introduction, structural variations (SVs) are rearrangements, relative to some reference genome sequence, of sequences of DNA with a length longer than 40 or 50bp. These can take the form of deletions of reference sequence, insertions of novel sequence, duplications of sequence, inversions of orientation of a stretch of sequence with respect to the rest of its chromosome, or translocations of sequence from one chromosome to another. Deletions, insertions, and duplications are also frequently referred to as \emph{copy number variations} (CNVs). In addition, many complex SVs exist in which these simple SV types are combined.

SVs can occur and become part of a gene pool on several different biological and time scales. On the smallest level, SVs can rearrange the genomes of individual cells within an organism. If the SV contributes to the cell's ability to proliferate, or occurs in conjuction with another mutation that does so, it can give rise to a tumor, or a new subpopulation or clone within an existing tumor. On a population scale, the genomes of individuals within a species harbor SVs with respect to one another, either because they were transmitted through the germline or because they arose \emph{de novo} in certain individuals. Finally, on a larger scale, SVs within a species may become fixed, contributing to differences between the genomes of varying species on an evolutionary scale.

\subsection{SV Effects on Phenotype and Disease}

On all of these levels, SVs are associated with differences in phenotypes and have been shown to associate with many types of disease, making them an important area of study. Knowledge of these effects has of course been shaped by the types and number of SVs that technology and collected data has allowed us to detect, and is therefore biased towards larger variants that were discoverable before the advent of modern array and sequencing technologies. 
Cancer genomes contain many SVs that have been linked to cancer progression. The famous Brc-Abl mutation~\cite{Kurzrock:2003bz} mentioned previously is the most famous example of a fusion gene, in which portions of the genome coding for two different genes are made adjacent due to an SV. This results in the creation of new gene which can take on novel functions, which in turn drive cancer growth. Fusion genes can be caused through any type of SV that causes a novel adjacency between regions of the genome, including deletions, duplications, translocations, and inversions, and there have been dozens of recurring fusion genes identified across varying cancers to this point~\cite{Annala:2013ks}. In another example, the genomic locus that contains the MYC gene, which among other functions regulates cell proliferation~\cite{Eilers:2008jk}, is frequently duplicated multiple times in breast cancer cells~\cite{Escot:1986tn}. These duplications change the amount of protein made by the cell, causing a dosage-related defect in cell regulation. In other cancer cells, a dramatic process called \emph{chromothripsis} has been observed, in which entire chromosomes appear to ``shatter'' and then be pieced back together. This gives rise to hundreds of SVs, some of which can promote cancer development~\cite{Stephens:2011bm}. On the other end of the spectrum in terms of variant size, small deletions of less than 150bp in untranslated coding regions of genes have been observed to contribute to abberant gene expression in leukemia~\cite{Hosokawa:1998wi}.

On the scale of the human population, SVs and CNVs have been implicated in a variety of diseases (for a comprehensive recent review see Weischenfeldt et al.~\cite{Weischenfeldt:2013fm}). The best studied of these are large CNVs, including deletions or duplications of genes. These can can include variants that arise \emph{de novo} in individuals, either giving rise to rare disorders~\cite{Lupski:1998ip} or contributing to the risk of more common conditions such as Crohn's disease~\cite{McCarroll:2008jt}, autism~\cite{Sebat:2007bs}, and schizophrenia~\cite{Walsh:2008kp}. However, most CNVs in the human genome are polymorphic within the population and contribute to human genetic diversity~\cite{McCarroll:2008p265}. The rise of microarrays and high-throughput sequencing has sped the discovery of smaller variants that affect coding sequence by deleting exons. To give one recent example, an 8kb deletion of an exon in the BAG3 gene has been tied to risk for cardiomyopathy~\cite{Norton:2011ev}. \todo{etc.}

There are also numerous examples of deletions that affect human phenotypes even though they do not directly overlap with the coding regions of genes, instead interacting with regulatory elements that control gene expression levels. For example, deletions ranging from 36kb to 319kb in size that are over one megabase away from the SOX9 gene locus can alter its expression by removing an enhancer element, giving rise to a rare cleft palate syndrom called Pierre Robin sequence~\cite{Benko:2009dq}. Expanding on this idea, a survey of CNVs between inbred mouse strains showed that 28\% of gene expression differences between strains could be statistically explaned by CNVs, but that only 7\% of the identified CNVs directly overlapped one of the differentially expressed genes, suggesting that SVs may shape many aspects of organism phenotypes in complex ways~\cite{Cahan:2009ef}. In another dramatic example from other species, the well-studied stickleback fish, which has repeatedly evolved adaptations to fresh water in several locations around the world as it moved from the sea into lakes, owes one of those adaptations, the loss of spines on its pelvis, to recurrent sub-2kb deletions of an enhacer that regulates the Pitx1 gene~\cite{Chan:2010hz}.

\todo{This last case points to the value that can be gained by studying SVs that have been shaped by, or helped to shape, evolutionary processes. Hahn gene duplications.. Ventura segmental duplications in gorilla.. Gibbon karyotype.. }

\subsection{Mechanisms and Signatures of SV Formation}

Given that many of the phenotypic effects of SVs described above are deleterious to the organism, it is natural to ask how they arise in genomes. Biologists have identified several different mechanisms that can cause SVs. By studying many of the genomic disorders mentioned in the previous section, geneticists found that often CNVs recur in different individuals at the same locations in the genome. Examining these locations revealed that they are home ot genomic sequence repeated elsewhere in the genome in the form of segmental duplications, stretches of DNA that are over 1kb in length with greater than 90\% sequence identity\cite{Sharp:2006fy}. This helped to identify the process of non-allelic homologous recombination (NAHR)~\cite{Liu:2012if}. In the best studied form of NAHR, the cell attempts to repair DNA in the cell that has suffered double-stranded breaks (DSBs). To do so, it attempts to find homologous DNA in the nucleus that can be used as a template to repair the broken strand. However, repeats in the genome provide the opportunity for DNA from the wrong genomic location to be used in this repair process, creating genomic breakpoints in the repaired DNA. In mammals, as little as 295bp of non-allelic homologous sequence can be responsible for NAHR events~\cite{Liskay:1987wt}. The human genome contains many copies of transposable elements, including 500,000 of the LINE family and over one million of the \emph{Alu} family, with average lengths of 6kb and 300bp, resepectively. Both \emph{Alu}~\cite{Lehrman:1985tn} and LINE~\cite{Robberecht:2013kw} are frequently involved in NAHR events. As these and other transposons copy themselves into new locations in the genome, they also give rise to their own class of SV, mobile element insertions (MEIs), which leave their own signatures of small tandem site duplications (TSDs) at either end of the insertions.

In contrast to NAHR, other mechanisms can create SVs without extensive sequence similarity at the breakpoint. Non-homologous end joining (NHEJ) and microhomology mediated end joining (MMEJ) are two such processes~\cite{Hastings:2009fv}. Both of these mechanisms involve the repair of DSBs in the cell and typically result in SV breakpoints that have very small (less than 20bp) homologies on either side. Other processes of SV formation that leave a signature of microhomologies at the breakpoint are Fork Stalling and Template Switching (FoSTeS)~\cite{Lee:2007fy} and microhomology-mediated break-induced replication (MMBIR)~\cite{Hastings:2009fv}. In these models, as the cell replicates all of its DNA two replication forks can switch DNA templates, causing complex structural variations to form. Finally, the chromothripsis process mentioned above also gives rise to microhomologies~\cite{Liu:2011p1751}.

Given that different SV formation mechanisms have different signatures that can be determined by the presence of varying amounts of duplication or homology at or near the breakpoints, it is possible to analyze SVs to gain insight into their formation in different contexts. For example, Conrad et al.~\cite{Conrad:2010if} surveyed CNVs from 40 unrelated individuals using a targeted sequencing method that enabled them to determine exact sequences around the breakpoints of some events. They found that although the best studied human genomic disorders are due to recurrent SVs caused by NAHR, NHEJ and MMEJ were responsible for the great majority of CNVs they were able to detect, a result later confirmed by the 1000 Genomes Project~\cite{Mills:2011p1611}, although studies in mice have found that MEI may be more frequent albeit harder to detect~\cite{Yalcin:2011tm}. \todo{cancer?} In Chapter~\ref{chap_breakpoint_analysis}, we will discuss efforts to analyze the sequence context of genomic rearrangements between species.

\section{High-Throughput Short-Read Sequencing}

The pace of analysis of genome rearrangements and SVs has risen dramatically with the widespread adoption of high-throughput short-read sequencing (for a review see Shendure and Ji~\cite{Shendure:2008jh}). In early projects to interrogate SVs in DNA samples researchers had to painstakingly map individual variants using time-consuming microscopy (for very large rearrangements), flouresence in situ hybridization (FISH), southern blotting, or polymerase chain reaction assays~\cite{Aten:2008dh}. The development of microarray technology allowed simultaneous testing of many genomic probes on a sample, although the probes had to be developed using prior knowledge of what variants to expect. Short-read sequencing, however, allows direct DNA sequence data to be collected in a relatively unbiased way from the entire genome, overcoming these limitations.

High-throughput short-read sequencing technology is currently dominated by Illumina, Inc. (San Diego, CA), and unless otherwise noted we will be referring to data generated by Illumina instruments when we refer to short-read or next-generation sequencing data in this document. The Illumina sequencing process begins with creation of genomic DNA library, which has billions of small fragments of DNA, sheared randomly across the genome, with adapter seqeuences attached to the ends. These adapters attach randomly to locations on the surface of the sequencing instrument's \emph{flow cell}, and then are amplified in place to form clusters on the flow cell, each consisting of thousands of copies of a DNA fragment from the library. The seqeuncer then initiates a process in which the complements of each strand of DNA in the clusters are synthesized one base at a time. Each complementary base added to the new molecule contains a flourescent tag, and the instrument takes a picture of the flow cell at each step. Pooling the color signal from each strand in a cluster, the sequencer optically analyzes the image from each time step to call the base at that position in each cluster. Because all clusters can be analyzed simultaneously in a single flow cell image, Illumina's sequencing instruments can now process billions of fragments at the same time. Illumina's HiSeq 2500 instrument, the current throughput leader, can produce 1.2 billion paired-end reads, or 120Gb of data, in 27 hours. 

While this technique produces unprecedented amounts of sequencing data, the downside is that read lengths are limited by the extent to which the synthesis process can be regulated and kept in sync across all of the clusters on the flow cell. In Illumina's current high-throughput instruments, this limits read lengths to 100bp or 150bp. This can be mitigated somewhat through techniques that sequence both ends of a DNA fragments, producing the paired-end reads referred to earlier. If the size of the fragment is approximately, the distance between the two sequenced ends can be estimated, as we will discuss in detail in Section~\ref{section_read_pair}. These paired-end techniques typically use fragments of between 250 and 500bp, although in some cases \emph{mate pair} fragment sequencing can produce reads from the ends of much larger fragments (6-8kb).

Because nearly half of the human genome is made up of repetitive elements, the analysis of the short reads produced by these sequencing technologies present a unique set of challenges~\cite{Treangen:2011p1810}. As we mentioned earlier, repetitive elements such as \emph{Alu} and LINE appear many times in the genome. If a read, or pair of reads, falls entirely within a single repetitive element, it will be impossible to determine the location in the genome it came from if there has not been enough sequence divergence of the repeat over evolutionary time to distinguish it from the others. Because many SV formation mechanisms depend on sequence homolgies at the breakpoints, they are particularly likely to fall into repeats in the genome, complicating the task of SV detection from short-read data.

In addition to making SV detection difficult, short-read sequencing -- by the unprecedented volume of data it produces -- also presents challenges to developing scalable infrastructure and algorithms for processing sequencing data in general. We will return to the problem of SV detection in the next chapter, and in the remainder of this chapter will give an overview of the applications of high-throughput sequencing data, and will discuss the computational challenges they represent.

\subsection{Sequencing Analysis Pipelines}\label{section_pipelines}

\subsubsection{DNA Resequencing}

de Pristo\cite{DePristo:2011fo}

\subsubsection{RNA-seq}

Olshack et al\cite{Oshlack:2010kr}

\subsubsection{\emph{de novo} Assembly}

A full discussion of \emph{de novo} assembly algorithms is beyond the scope of this section.

Review\cite{Nagarajan:2013cq}

\subsection{Dealing with Big Data}

``Cloud computing and the DNA data race''\cite{Schatz:2010js}
``The case for cloud computing in genome informatics''\cite{Stein:2010gp}
``Translational Biomedical Informatics in the Cloud: Present and Future''\cite{Chen:2013ci} Collaboration...
``Computational solutions to large-scale data management and analysis''\cite{Schadt:2010dp} Cloud and Heterogenous computing enviros
\todo{maybe... ``Critical role of bioinformatics in translating huge amounts of next-generation sequencing data into personalized medicine''\cite{Hong:2013ev}}

\todo{maybe \cite{Stein:2008gh}}

GPU stuff
ISAAC aligner~\cite{Raczy:2013hy}